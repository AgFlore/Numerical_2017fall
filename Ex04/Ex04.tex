% !TEX TS-program = Xelatex
% !TEX encoding = UTF-8 Unicode

\documentclass[UTF8,size=9.5]{ctexart}
\usepackage{amsmath}
\usepackage{dsfont}
\usepackage[table]{xcolor}
\usepackage[bottom]{footmisc}
\usepackage{graphicx}
\usepackage{enumitem}
\usepackage{figsize}
\usepackage{standalone}
\usepackage[separate-uncertainty = true,tight-spacing=true,round-minimum=0.00000000001]{siunitx}
\usepackage{tabu}
\usepackage{wasysym}
\usepackage{geometry}
\geometry{left=0.7in,right=0.7in,bottom=0.7in,top=0.7in}

\title{计算物理学作业4}
\author{朱寅杰 1600017721}
\begin{document}

\maketitle

\section{利用标准的 4 阶 Runge-Kutta 来求解Lotka–Volterra方程的初值问题}
对于微分方程组
$\begin{cases}
	\dot{x}=\alpha x-\beta xy \\
	\dot{y}=\delta xy-\gamma y
\end{cases}$
,作变量代换
$\begin{cases}
	X=\delta x/\alpha\\
	Y=\beta y/\alpha\\
	T=\alpha t
\end{cases}$
,得到
$\begin{cases}
\dot{X}=X(1-Y)\\
\dot{Y}=Y(X-\gamma/\alpha),\gamma/\alpha>0
\end{cases}$
。对于这个方程组,我们首先考察其不含时的特解(即$\dot{X}=\dot{Y}=0$的解):其一为$X(T)=Y(T)=0$,其一为$X(T)=\gamma/\alpha,Y(T)=1$。作一个微扰的话,
\begin{enumerate}[label={\alph*)},font=\bfseries]
\item 对于前者,设$X(T)=\epsilon_1(T)$,$Y(T)=\epsilon_2(T)$,其中$\epsilon$均为小量,取一阶近似得在$T$很小时有$\begin{cases}\dot{\epsilon_1}=\epsilon_1\\\dot{\epsilon_2}=-\gamma/\alpha\epsilon_2(T))\end{cases}$,故$\begin{cases}\epsilon_1(T)\approx\epsilon_1(0)\exp{T}\\\epsilon_2(T)\approx\epsilon_2(0)\exp{(-T\gamma/\alpha)}\end{cases}$,也就是说对于$X$方向上的微扰这个不动点是不稳定的,而对于$Y$方向上的微扰这个不动点是稳定的。
\item 对于后者,取$X(T)=\gamma/\alpha+\epsilon_1(T),Y(0)=1+\epsilon_2(T)$,则有$\begin{cases}
                                                                         \dot{\epsilon_1}=-(\epsilon_1+\gamma/\alpha)\epsilon_2\\
                                                                         \dot{\epsilon_2}=(1+\epsilon_2)\epsilon_1
                                                                       \end{cases}$
再对$T$求导,保留至一阶小量,得到$\begin{cases}
                  \ddot{\epsilon_1}=-\epsilon_1(T)\times\gamma/\alpha\\
                  \ddot{\epsilon_2}=-\epsilon_2(T)\times\gamma/\alpha
                \end{cases}$
                故这是一个稳定不动点。
\end{enumerate}
事实上,这个微分方程组是可积的。将两式相除得到$\frac{\mathrm{d}Y}{\mathrm{d}X}=\frac{Y(X-\gamma/\alpha)}{X(1-Y)}$,分离变量积分得到$X+Y-\log{(X^{\gamma/\alpha}Y)}=$常数。

\end{document}


